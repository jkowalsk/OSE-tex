\chapter{Introduction}
Ce module est destiné
\begin{enumerate}
  \item Aux MJ expérimentés avec des joueurs débutants ;
  \item Aux MJ souhaitant apprendre à concevoir un donjon ;
  \item Aux MJ expérimentés avec des joueurs vétérans, mais découvrant l'OSR.
\end{enumerate}
Si vous êtes un MJ complètement débutant, vous pouvez toujours utiliser ce donjon et en apprendre beaucoup, mais il
mettra vos compétences à rude épreuve d'entrée de jeu.
Les joueurs expérimentés pourront probablement l'apprécier eux aussi.

\section{Taille du Groupe et Équilibre}
Ce donjon est conçu pour des personnages de niveau 1.
Vous pouvez tout aussi bien mener ce donjon pour un joueur comme pour dix.
Les rencontres ne sont pas équilibrées.
Elles n'ont pas de "facteur de puissance".
Le combat n'est que peu récompensé, mais l'exécution réussie d'un bon plan l'est grandement.

En fonction du style de jeu, de la vitesse de progression, des aventures annexes, du temps passé en ville et autres digressions, l'exploration exhaustive de ce donjon peut se compléter en 12 à 24 heures de jeu.

La durée du Niveau 1 est adaptée à une courte première session précédée d'une création de personnages.

\section{Appâter les PJ}
Voici quelques moyens d'amener les PJ au donjon, pour peu qu'ils soient sur la paille et qu'ils sachent que les tombes
contiennent souvent des trésors.
Vous pouvez placer ce donjon n'importe où.

\begin{enumerate}
  \item Ils trouvent une vieille carte menant à une sépulture tombée dans l'oubli.
  \item Un glissement de terrain révèle l'entrée de la tombe.
  \item Les gobelins (p. 13) kidnappent un proche des PJ.
  \item Les expériences de Xiximantre (p. 13) induisent en eux d'étranges rêves.
  \item Ils tombent sur l'entrée du tombeau alors qu'ils s'occupent d'un problème qui n'a rien à voir.
  \item Un puissant commanditaire les envoie explorer la tombe nouvellement découverte.
\end{enumerate}