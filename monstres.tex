\chapter{Description des monstres}
\section{Niveau 1}
\subsection{Squelette}\label{monster:s6}
\begin{itemize}
  \item \textbf{Apparence :} 
  \begin{itemize}
    \item Squelette humain muni de crocs et d’une arme rouillée. 
    \item Couvert de bracelets.
  \end{itemize}
  \item \textbf{ Désirs :} protéger le reste de la Tombe des Rois Serpents et tuer tout intrus.
\end{itemize}

\SetTblrInner{hspan=minimal, vspan=even}
\begin{osetable}{X[2]X[1]X[1]X[2]}{0}
  \SetCell[c=4]{bg=black,fg=white} {\bfseries\large\sectionfont Statistiques} & & &\\
  \textbf{CA}          & 7 [12] & \textbf{TACO}        & 19 [0] \\
  \textbf{Attaque}     & \SetCell[c=3]{l} 1D6 (griffes) & &\\
  \textbf{Sauvegardes} & \SetCell[c=3]{l} {\small \textbf{MP}~12 \textbf{B}~13 \textbf{PP}~14 \textbf{S}~15 \textbf{SSB}~16}& &\\
  \textbf{Mouvement} & 18m    & \textbf{Moral} & 12 \\
  \textbf{DV} & 1   & \textbf{XP} & 10 \\
  \textbf{PV} (\hspace*{20pt}) & \SetCell[c=3]{l}\noindent\hdsquares{1} & &\\
  \textbf{PV} (\hspace*{20pt}) & \SetCell[c=3]{l}\noindent\hdsquares{1} & &\\
  \textbf{PV} (\hspace*{20pt}) & \SetCell[c=3]{l}\noindent\hdsquares{1} & &\\
\end{osetable}

\begin{itemize}
  \item Ne subissent que la moitié des dégâts infligés par les armes perçantes ou tranchantes.
  \item Ils craquent et cliquettent, meurtriers et implacables
\end{itemize}


\vfill
%\pagebreak
\section{Niveau 2}
\subsection{Fragments de Momie}\label{monster:s11}
\begin{itemize}
  \item \textbf{Apparence :} 
  \begin{itemize}
    \item Bras noircis et décomposés
    \item Doigts acérés
  \end{itemize}
  \item \textbf{ Désirs :}  étrangler des choses, occire les vivants.
\end{itemize}

\SetTblrInner{hspan=minimal, vspan=even}
\begin{osetable}{X[2]X[1]X[1]X[2]}{0}
  \SetCell[c=4]{bg=black,fg=white} {\bfseries\large\sectionfont Statistiques} & & &\\
  \textbf{CA}          & 7 [12] & \textbf{TACO}        & 19 [0] \\
  \textbf{Attaque}     & \SetCell[c=3]{l} 1D6 (griffes) & &\\
  \textbf{Sauvegardes} & \SetCell[c=3]{l} {\small \textbf{MP}~12 \textbf{B}~13 \textbf{PP}~14 \textbf{S}~15 \textbf{SSB}~16}& &\\
  \textbf{Mouvement} & 18m    & \textbf{Moral} & 12 \\
  \textbf{DV} & 1*   & \textbf{XP} & 13 \\
  \textbf{PV} (\hspace*{20pt}) & \SetCell[c=3]{l}\noindent\hdsquares{1} & &\\
  \textbf{PV} (\hspace*{20pt}) & \SetCell[c=3]{l}\noindent\hdsquares{1} & &\\
\end{osetable}

\begin{itemize}
\item \textbf{Maladie :}
\begin{itemize}
  \item Maladie putrescente au toucher
  \item Guérison magique inefficace
  \item Guérison naturelle 10x plus longue
  \item Ne peut être éliminée que par magie
\end{itemize}
\item \textbf{Mort-vivant}
\begin{itemize}
  \item Silencieux. 
  \item Insensible aux effets affectant les créatures vivantes. 
  \item Immunisée contre les sorts affectant ou lisant l’esprit.
\end{itemize}
\end{itemize}

Ils se tortillent, grimpent votre corps et tentent de vous étrangler.

\vfill
\pagebreak

\subsection{Sparamantur}\label{monster:s13}
\begin{itemize}
  \item \textbf{Apparence :} 
  \begin{itemize}
    \item Squelette humain muni de crocs et d’une arme rouillée. 
    \item Couvert de bracelets.
  \end{itemize}
  \item \textbf{ Désirs :} protéger le reste de la Tombe des Rois Serpents et tuer tout intrus.
\end{itemize}

\SetTblrInner{hspan=minimal, vspan=even}
\begin{osetable}{X[2]X[1]X[1]X[2]}{0}
  \SetCell[c=4]{bg=black,fg=white} {\bfseries\large\sectionfont Statistiques} & & &\\
  \textbf{CA}          & 7 [12] & \textbf{TACO}        & 17 [2] \\
  \textbf{Attaque}     & \SetCell[c=3]{l} 1D8 (Hache de bataille) & &\\
  \textbf{Déplacement} & 18m    & \textbf{Moral} & 12 \\
  \textbf{Sauvegardes} & \SetCell[c=3]{l} {\small \textbf{MP}~12 \textbf{B}~13 \textbf{PP}~14 \textbf{S}~15 \textbf{SSB}~16}& &\\
  \textbf{Mouvement} & 18m    & \textbf{Moral} & 12 \\
  \textbf{DV} & 3   & \textbf{XP} & 35 \\
  \textbf{PV} (\hspace*{20pt}) & \SetCell[c=3]{l}\noindent\hdsquares{3} & &\\
\end{osetable}

\begin{itemize}
  \item Ne subit que la moitié des dégâts infligés par les armes perçantes ou tranchantes.
  \item Il craque et cliquette, meurtrier et implacable
\end{itemize}

\vfill
\pagebreak

\subsection{Pudding Noir (Franbinzar)}\label{monster:s14}
\begin{itemize}
  \item \textbf{Apparence :}  100 kg de mélasse noire
  \item \textbf{ Désirs :} Se nourrir.
\end{itemize}

\begin{osetable}{X[2]X[1]X[1]X[2]}{0}
 \SetCell[c=4]{bg=black,fg=white} {\bfseries\large\sectionfont Statistiques} & & &\\
 \textbf{CA}          & 6 [13] & \textbf{TACO}        & 15 [4] \\
 \textbf{Attaque}     & \SetCell[c=3]{l} 2D8 (toucher) & &\\
 \textbf{Sauvegardes} & \SetCell[c=3]{l} {\small \textbf{MP}~10 \textbf{B}~11 \textbf{PP}~12 \textbf{S}~13 \textbf{SSB}~14}& &\\
 \textbf{Mouvement} & 18m    & \textbf{Moral} & 12 \\
 \textbf{DV} &  5   & \textbf{XP} & 300 \\
 \textbf{PV} (\hspace*{20pt}) & \SetCell[c=3]{l}\noindent\hdsquares{5} & &\\
\end{osetable}

\begin{itemize}
  \item \textbf{Immunité :} Blessé uniquement par le feu, autres: dégâts temporaires
  \item \textbf{Division :} Les attaques qui ne sont pas liées au feu (y compris les sorts) provoquent la division du pudding. Chaque coup crée un nouveau pudding de 1 DV qui inflige 1d6 points de dégâts.
  \item \textbf{Corrosion :} Peut dissoudre le bois ou le métal en un tour (10\% de chances).
  \item \textbf{Collant :} Peut se déplacer sur les murs et les plafonds.
  \item \textbf{Suintement :} Peut s’infiltrer à travers les petits trous et fissures.
  \item \textbf{Régénération :} Si tué, se régénère en 1d20 heures, à moins d’être incinéré.
  \item \textbf{Attaques multiples :} Peut cibler tous les PJ adjacents à chaque round, effectuant un jet d’attaque classique pour chacun.
\end{itemize}

Franbinzar était le dernier roi de la forteresse.

Sa momification ne s’est pas déroulée correctement.
Il s’est transformé en Pudding Noir.

On peut apercevoir 200PO d’anneaux noyés dans sa masse.

\vfill
\pagebreak

\subsection{Gardien Cobra de Pierre}\label{monster:s19}
\begin{itemize}
  \item \textbf{Apparence :} 
  \begin{itemize}
    \item  Chevalier de pierre vêtu d’une armure sculptée. 
    \item Une de ses mains manie une imposante épée dentelée.
    \item L’autre est libre au début du combat.
  \end{itemize}
  \item \textbf{ Désirs :}  protéger le reste de la Tombe des Rois Serpents et  tuer les intrus.
\end{itemize}

\begin{osetable}{X[2]X[1]X[1]X[2]}{0}
  \SetCell[c=4]{bg=black,fg=white} {\bfseries\large\sectionfont Statistiques} & & &\\
  \textbf{CA}          & 3 [16] & \textbf{TACO}        & 15 [4] \\
  \textbf{Attaque}     & \SetCell[c=3]{l} 2D6 (épée dentelée) & &\\
  \textbf{Sauvegardes} & \SetCell[c=3]{l} {\small \textbf{MP}~10 \textbf{B}~11 \textbf{PP}~12 \textbf{S}~13 \textbf{SSB}~14}& &\\
  \textbf{Mouvement} & 18m    & \textbf{Moral} & 12 \\
  \textbf{DV} &  4+1   & \textbf{XP} & 125 \\
  \textbf{PV} (\hspace*{20pt}) & \SetCell[c=3]{l}\noindent\hdsquares{4}~$\square$ & &\\
\end{osetable}

\textbf{Attaques :} Chaque round, le Gardien Cobra de Pierre peut
appliquer une de ces stratégies :
\begin{enumerate}
  \item \textbf{Appel du Bouclier :} 
  Le Gardien appelle à lui un des boucliers couvrant les murs de l’arène.
  \begin{itemize}
    \item 1d6 dégâts (jet de sauvegarde contre la mort pour esquiver) si sur trajectoire
    \item Garde le bouclier (+1 CA)
  \end{itemize}
  \item \textbf{Bond et Impact :} 
  Le Gardien bondit à travers les airs puis plonge jusqu’à 6 m de sa position initiale.
  \begin{itemize}
    \item toute créature adjacente : 1d4 dégâts (sauvegarde contre la mort annule). 
    \item si dégât : projeté au sol
  \end{itemize}
  \item \textbf{Attaque normale}
\end{enumerate}

\textbf{Stratégies possibles :}
\begin{itemize}
  \item Prendre le Gardien en \textbf{tenaille}
  \item \textbf{Fuir :} Trop imposant pour l'escalier
  \item \textbf{Courir :} suit le PJ tant que détectables
\end{itemize}

\section{Niveau 3}
\subsection{Monstres Errants}\label{monster:n3:errants}
\begin{enumerate}
  \item \textbf{Présage du Basilic.} 
  Des cliquetis et grincements d’une chaîne distante traînée sur de la roche et dans la poussière.
  \item \textbf{Présage de Gelées Squelettes.} 
  Des bruits de succion résonnant dans le lointain.
  \item \textbf{Présage de Gobelins.} 
  \begin{itemize}
    \item Chuchotis, gloussements, grincements de dents et pourléchage de babines. 
    \item Le reflet d’yeux rouges dans l’obscurité. 
    \item Un fumet de pourriture fongique.
  \end{itemize}  
  \item \textbf{Chauve-souris.} 
  Nullement hostile, mais effrayante sur le moment. 
  Volette de-ci de-là, plane en direction du gouffre.
  \item \textbf{\nameref{monster:n3:araignee}} 
  \item \textbf{1d6 \nameref{monster:n3:gob}} en reconnaissance. 
  \begin{itemize}
    \item 1d6 autres gobelins fongiques postés au détour du prochain couloir.
  \end{itemize}
  \item  \textbf{1 \nameref{monster:n3:squelgel}.}
  \item  \textbf{1d10+5 \nameref{monster:n3:gob}} parés au combat. 
  L’un d’eux est muni d’une lance faite de couverts, absurdement peu pratique à manier (1d6 dégâts, allonge)
\end{enumerate}

\vfill
\pagebreak
\subsection{Grosse Araignée}\label{monster:n3:araignee}
  \begin{itemize}
  \item De la taille d’un poing.
  \item Cherche à manger des chauves-souris, pas les PJ.
  \item \textbf{Venimeuse : } 1d4 dégâts (sauvegarde poison annule) 
  \item \textbf{lâche}
  \item Mets de choix pour les gobelins fongiques.
\end{itemize} 

\SetTblrInner{hspan=minimal, vspan=even}
\begin{osetable}{X[2]X[1]X[1]X[2]}{0}
  \SetCell[c=4]{bg=black,fg=white} {\bfseries\large\sectionfont Statistiques} & & &\\
  \textbf{CA}          & 7 [12] & \textbf{TACO}        & 19 [0] \\
  \textbf{Attaque}     & \SetCell[c=3]{l} 1D3 (morsure) + poison & &\\
  \textbf{Sauvegardes} & \SetCell[c=3]{l} {\small \textbf{MP}~12 \textbf{B}~13 \textbf{PP}~14 \textbf{S}~15 \textbf{SSB}~16}& &\\
  \textbf{Mouvement} & 9m    & \textbf{Moral} & 8 \\
  \textbf{DV} & 1/2   & \textbf{XP} & 5 \\
  \textbf{PV} (\hspace*{20pt}) & \SetCell[c=3]{l}\noindent$\square\square\square\square$ & &\\
\end{osetable}

\subsection{Gobelins Fongiques}\label{monster:n3:gob}
\SetTblrInner{hspan=minimal, vspan=even}
\begin{osetable}{X[2]X[1]X[1]X[2]}{0}
  \SetCell[c=4]{bg=black,fg=white} {\bfseries\large\sectionfont Statistiques} & & &\\
  \textbf{CA}          & 6 [13] & \textbf{TACO}        & 19 [0] \\
  \textbf{Attaque}     & \SetCell[c=3]{l} 1D6 (arme) & &\\
  \textbf{Sauvegardes} & \SetCell[c=3]{l} {\small \textbf{MP}~14 \textbf{B}~15 \textbf{PP}~16 \textbf{S}~17 \textbf{SSB}~18}& &\\
  \textbf{Mouvement} & 18m    & \textbf{Moral} & 7 \\
  \textbf{DV} & 1-1   & \textbf{XP} & 5 \\
  \textbf{PV} (\hspace*{20pt}) & \SetCell[c=3]{l}\noindent\hdsquares{1} & &\\
  \textbf{PV} (\hspace*{20pt}) & \SetCell[c=3]{l}\noindent\hdsquares{1} & &\\
  \textbf{PV} (\hspace*{20pt}) & \SetCell[c=3]{l}\noindent\hdsquares{1} & &\\
  \textbf{PV} (\hspace*{20pt}) & \SetCell[c=3]{l}\noindent\hdsquares{1} & &\\
  \textbf{PV} (\hspace*{20pt}) & \SetCell[c=3]{l}\noindent\hdsquares{1} & &\\
  \textbf{PV} (\hspace*{20pt}) & \SetCell[c=3]{l}\noindent\hdsquares{1} & &\\
  \textbf{PV} (\hspace*{20pt}) & \SetCell[c=3]{l}\noindent\hdsquares{1} & &\\
\end{osetable}

\begin{itemize}
  \item \textbf{Apparence :} 
  \begin{itemize}
    \item pâles et rabougris, 
    \item \'Enorme tête ovale
    \item Plein de dents et deux petits yeux rouges beaucoup trop près l’un de l’autre. 
    \item Texture évoque une purée de pommes de terre à la colle blanche. 
    \item Portent sur eux des couverts de  table.
  \end{itemize}
  \item \textbf{ Désirs :} un roi, à manger, des choses qui brillent, encore à manger
  \item \textbf{Non hostiles :} ils cherchent seulement à couronner quelqu’un Roi des Gobelins (Initialement)
  \item \textbf{Dialecte :} gobelin cliquetant et limité.
  \item Aisément \textbf{corruptibles}
  \item Si PJ hostiles:
  \begin{itemize}
    \item \textbf{Fuite}
    \item \textbf{Embuscades} récurrentes
    \item Sournois et patients
    \item Escaladent (lentement) les murs : surprendre les aventuriers en leur plongeant dessus
    \item Seaux d’eau pour éteindre les torches
    \item Cordes pour emmêler les combattants
    \item Tirent profit des pièges du donjon
    \item De nuit, harcèlent le campement
  \end{itemize}
\end{itemize}

Ils suivent loyalement leur roi jusqu’à la pleine lune, puis l’assaillent, le traînent jusqu’à un autel au sommet d’une colline et l’éventrent. 

Ils tentent d’avertir le groupe au sujet du Basilic (p. 14), mais ignorent l’existence du 39 : Passage Secret (p. 9) et ne connaissent rien des niveaux supérieurs : le Gardien Cobra de Pierre leur empêche d’y accéder. 

La nuit venue, ils utilisent les 41 : Escaliers vers la Surface (p. 9) pour se faufiler jusqu’à l’extérieur. 

À moins que la 48 : Fosse de Génération de Gobelins (p. 11) ne soit incendiée, le nombre de gobelins dans le donjon sera toujours "beaucoup trop de gobelins". 
Les gobelins fongiques sont des rats de laboratoire parvenus à s’échapper.

Bien que Xiximantre (p. 13) ne s’oppose pas à ce qu’on les lui rende, ils ne lui sont pas d’une grande utilité.

\subsection{Gelée Squelette}\label{monster:n3:squelgel}
\begin{itemize}
  \item \textbf{Apparence :} Un squelette couvert de mucus orange
  \item \textbf{ Désirs :}  écraser des crânes et créer plus de gelées squelettes.
\end{itemize}

\SetTblrInner{hspan=minimal, vspan=even}
\begin{osetable}{X[2]X[1]X[1]X[2]}{0}
  \SetCell[c=4]{bg=black,fg=white} {\bfseries\large\sectionfont Statistiques} & & &\\
  \textbf{CA}          & 7 [12] & \textbf{TACO}        & 19 [0] \\
  \textbf{Attaque}     & \SetCell[c=3]{l} 1D4 (griffes) & &\\
  \textbf{Sauvegardes} & \SetCell[c=3]{l} {\small \textbf{MP}~12 \textbf{B}~13 \textbf{PP}~14 \textbf{S}~15 \textbf{SSB}~16}& &\\
  \textbf{Mouvement} & 9m    & \textbf{Moral} & 12 \\
  \textbf{DV} & 2   & \textbf{XP} & 20 \\
  \textbf{PV} (\hspace*{20pt}) & N.A. & &\\
  \textbf{PV} (\hspace*{20pt}) & N.A. & &\\
  \textbf{PV} (\hspace*{20pt}) & N.A. & &\\
  \textbf{PV} (\hspace*{20pt}) & N.A. & &\\
\end{osetable}

\begin{itemize}
  \item Immunisés à tous dommages.
  \item \textbf{Solutions possibles:}
  \begin{itemize}
    \item Fuir, 
    \item les faire pétrifier par le Basilic (p. 14), 
    \item les précipiter dans le gouffre, 
    \item les attacher
    \item les enfermer dans une salle
    \item les précipiter en 25 : Fosse Piégée (p. 6) ou 37 :  Fosse Piégée (p. 9)
  \end{itemize}
\end{itemize}

Il y a 4 gelées squelettes dans le donjon. 
Si le groupe les immobilise toutes les 4, retirez-les de la \textbf{Table des \nameref{monster:n3:errants}}. 

Au bout d’un certain temps, ils finiront toujours par sortir des fosses et se libérer des cordes. 

Toute créature vivante tuée par une gelée squelette se transforme en une nouvelle gelée squelette en 10 minutes. 
Les gobelins fongiques sont immunisés.